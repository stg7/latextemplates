\documentclass[
    ngerman,
    a4paper,
    %twoside, % für zweiseiten druck
    %openright,
    %cleardoublepage=plain,
    headsepline,
    footsepline,
    chapterprefix, % Für Kapitel X \\ Kapitelname
    numbers=noenddot,
    12pt,
    fleqn,
    bibliography=totoc, % literaturverzeichnis zum inhaltsverzeichnis
    listof=totoc % alle anderen verzeichnisse auch zum inhaltsverzeichnis hinzufügen
    ]{scrreprt}

% packages
\usepackage[utf8]{inputenc}

% fülltext spielerein
\usepackage{lipsum}
\usepackage{blindtext}

\usepackage[normalem]{ulem}
\usepackage{fancybox}

\usepackage{graphicx}
\usepackage[usenames,dvipsnames]{xcolor}
\usepackage{url}

\usepackage{listings}

\usepackage{longtable}
\usepackage{tabularx}       % Fuer Tabellen laenger als eine Seite
\usepackage{multicol,multirow}
\usepackage{colortbl} % farbige tabellen

\usepackage{latexsym}
\usepackage{amsmath}
\usepackage{float,placeins}

\renewcommand{\labelitemi}{$\triangleright$}
\renewcommand{\labelitemii}{$\diamond$}
\renewcommand{\labelitemiii}{$\circ$}

% Definitonen
% begin{definition} blabla bla \end{defitiniton}
\newtheorem{definition}{Definition}

\usepackage{amssymb}


\usepackage[ngerman]{babel}
\usepackage[automark]{scrpage2}

% algos
\usepackage[german, vlined,linesnumbered,boxruled]{packages/algorithm2e}
%\usepackage[vlined,linesnumbered,boxed]{packages/algorithm2e}

\usepackage{qtree} % für Bäume


% latex pictures
\usepackage{tikz}
\usetikzlibrary{shapes,snakes,arrows,matrix,calc,er}
\usepackage{pgfplots}

%\usetikzlibrary{decorations,markings}
\usepackage{booktabs}

% Index erzeugen
%\usepackage{scrindex} \usepackage{makeidx}
%\makeindex{}

\usepackage{csquotes} % deutsche Anführungszeichen durch "` und  "'

%\usepackage[colorinlistoftodos,disable]{todonotes} % todonotes disable
\usepackage[colorinlistoftodos]{todonotes} % todonotes

\usepackage{caption}
\usepackage{subcaption}

\usepackage{packages/wrapfig}

% farbdefinitionen
\definecolor{lightblue}{rgb}{0,0.6,1.0}
\definecolor{lightgray}{rgb}{0.2,0.2,0.2}
\definecolor{blue}{rgb}{0,0.3,0.6}
\definecolor{darkblue}{rgb}{0,0.1,0.4}
\definecolor{green}{rgb}{0,0.7,0.2}



% Paket für Links innerhalb des PDF Dokuments
\usepackage[%
	%pdftitle={\thema{}},% Titel der erzeugten PDF Datei
	%pdfauthor={\author{}},%
	%pdfcreator={LaTeX Skript by stg7},
	%pdfsubject={\thema{}},%
	%pdfkeywords={\schluesselwoerter{}}
	bookmarksopen=true,%
	bookmarksopenlevel=1,%
	plainpages=false%
	]{hyperref}

\hypersetup{%
	colorlinks=true,% links einfaerben, oder box drum malen?
	linkcolor={red},% verweise im doc, ua inhaltsverzeichnis
	citecolor={blue},
	filecolor={black},
    filecolor=blue,
	urlcolor={green}
    }

% Kopf- und Fußzeile
\clearscrheadfoot
\chead{\headmark} % automatischen Kapitelnamen rein
\ofoot[\pagemark]{\pagemark} % oben rechts Seitenzahl laut Richtlinie
%\ifoot{\inventarisierungsnr}
\pagestyle{scrheadings}

% --------------------

% Seitendefinitionen

\setlength{\topmargin}{0.5cm}
\setlength{\headheight}{12pt}
\setlength{\headsep}{10pt}
\setlength{\topskip}{12pt}
\setlength{\evensidemargin}{0pt}
\setlength{\oddsidemargin}{0pt}
\setlength{\textheight}{240mm}
\setlength{\textwidth}{160mm}
\setlength{\voffset}{-2cm}
\setlength{\parindent}{0pt}
\setlength{\parskip}{6pt}


% Schrifteinstellungen
\usepackage[T1]{fontenc}
\usepackage{lmodern} % moderne schriftart
\renewcommand*\familydefault{\sfdefault}

\setkomafont{sectioning}{\normalfont\bfseries}
\setkomafont{descriptionlabel}{\normalfont\bfseries}
\setkomafont{captionlabel}{\bfseries\footnotesize}
\setkomafont{caption}{\footnotesize}

% schriftfarbe für überschriften

\addtokomafont{chapter}{\color{darkblue}}
\addtokomafont{section}{\color{darkblue}}
\addtokomafont{subsection}{\color{blue}}
\addtokomafont{subsubsection}{\color{lightgray}}
\addtokomafont{paragraph}{\color{blue}}

% für dictum
\renewcommand*{\dictumwidth}{.5\textwidth}
\renewcommand*{\dictumauthorformat}[1]{\textsc{#1}\bigskip}


% eigene commands
\newcommand{\onecm}{\hspace{1cm}}
\newcommand{\hide}[1]{}

\newcommand{\whide}[1]{\textcolor{white}{#1}}

% Abbildungsreferenz
\newcommand{\picref}[1]{Abbildung~\ref{#1}}

% eventuelles Umbenennen des Literaturverzeichnisses in Quellen[verzeichnis]
%\renewcommand{\bibname}{Quellen}

% rausblenden
\usepackage{ifthen}
\newboolean{includethis}

\newcommand{\zusatz}{}
%\setboolean{includethis}{false} % abgabe version
\setboolean{includethis}{true} % mit allem
\newcommand{\ifinclude}[1]{\ifthenelse{\boolean{includethis}}{#1}{}}


\usepackage{setspace}
\singlespacing

\usepackage{enumitem}
\usepackage[printonlyused,nohyperlinks]{acronym} % abkürzungen

\newcommand{\todoI}[1]{\todo[inline]{#1}}
\newcommand{\note}[1]{\todo[inline,color=green!40]{#1}}
\newcommand{\last}[1]{\todo[inline,color=blue!40]{#1}}

\newcommand{\G}[1]{{ \color{green!45!black}#1}}
\newcommand{\R}[1]{{ \color{red!45!black}#1}}
\newcommand{\B}[1]{{ \color{blue!70!black}#1}}
\newcommand{\W}[1]{{ \color{white}#1}}

%listing setup
\lstset{tabsize=4,
    basicstyle=\footnotesize\ttfamily,
    stringstyle=\color{Orange},
    showstringspaces=false,
    columns=fixed,
    numberstyle=\tiny\ttfamily,
    numbersep=15pt,
    commentstyle=\color{LimeGreen},
    language=[GNU]C++,
    identifierstyle=\color{MidnightBlue}}

\lstset{
    numbers=right,
    xrightmargin=2em,
    breaklines=true,
    breakatwhitespace=true,
    prebreak=\mbox{\tiny$\searrow$},
        postbreak=\mbox{{\color{blue}\tiny$\rightarrow$}},
    numberbychapter=false,
}

% zentrierte Bilder in Tabellen aus l2picfaq.pdf
% neuer Befehl: \includegraphicstotab[..]{..}
% Verwendung analog wie \includegraphics
\newlength{\myx} % Variable zum Speichern der Bildbreite
\newlength{\myy} % Variable zum Speichern der Bildhöhe
\newcommand\includegraphicstotab[2][\relax]{%
    % Abspeichern der Bildabmessungen
    \settowidth{\myx}{\includegraphics[{#1}]{#2}}%
    \settoheight{\myy}{\includegraphics[{#1}]{#2}}%
    % das eigentliche Einfügen
    \parbox[c][1.1\myy][c]{\myx}{%
    \includegraphics[{#1}]{#2}}%
}% Ende neuer Befehl
\newcommand{\Happy}{\includegraphicstotab[width=15px]{images/smiley/happy}}
\newcommand{\Okay} {\includegraphicstotab[width=15px]{images/smiley/alien}}
\newcommand{\Bad}  {\includegraphicstotab[width=15px]{images/smiley/bad}}


% für später nur pdf grafiken, anstelle der png
\DeclareGraphicsExtensions{.pdf,.png,.jpg}
%\renewcommand{\B}[1]{{ \color{white}#1}}

\usepackage[all]{nowidow}
