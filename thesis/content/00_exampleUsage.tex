\chapter{Benutzung der Vorlage}

Das vorliegende Dokument dient als Vorlage .

Im folgenen soll ein kurzer Überblick über die Möglichkeiten von Latex gegeben werden.

Dies hier ist ein Bibtex-Verweis \cite{sparqlAlgebra}.
Und gleich noch einer \cite{coresparql}. Und noch einer \cite[Kapitel X, Seit XY ff]{sparqlAlgebra}

Für bibtex einträge kann man verschiedene Quellen benutzen,
beispielsweise bietet
\url{http://scholar.google.de/} Bibtex Export (unter einstellungen umstellen),
oder \url{http://books.google.de} , oder \url{http://citeseerx.ist.psu.edu}
oder \url{http://ieeexplore.ieee.org/}

\section{Aufzählungen}

Aufzählungen als Anstriche
\begin{itemize}
    \item Punkt 1
    \item Punkt 2
        \begin{itemize}
            \item Punkt 1
            \item Punkt 2
                \begin{itemize}
                    \item Punkt 1
                    \item Punkt 2
                \end{itemize}
        \end{itemize}
\end{itemize}

oder numeriert
\begin{enumerate}
    \item Punkt 1
    \item Punkt 2
\end{enumerate}

\section{Tabellen}

Tabellen in Latex, einfacher gehts nimmer:

\begin{center}
    \begin{tabular}{|l|cc|}
        \hline
        \hline
        a & b & c \\
        \hline
        \hline
        a & y  & z \\
        \hline
        1 & 2 & 3 \\
        \hline
        \hline
    \end{tabular}
\end{center}
Besser Tabelle~\ref{tbl:smartptrTime} (Tabellen in Büchern haben zur besseren Optik keine vertikalen Linien!):
\begin{table}[ht]
    \centering

    \caption{\% Zeitbedarf für Smart-Pointer/Raw-Pointer im Projekt}
    \begin{tabular}{lrrrr}
    \toprule
    ~ &  Smart-Pointer (boost) & raw &  & \\
    \midrule
    builddb &  $\approx 42\%$ & $\approx 4.2\%$ & &  \\
    \midrule
    Q1 & $\approx 35\%$ & $\approx 3.5\%$ & & \\
    Q2 & $\approx 40\%$ & $\approx 4\%$ & & \\
    Q3 & $\approx 0.5\%$& $\approx 0.05\%$ & & \\
    Q4 & $\approx 36\%$& $\approx 3.6\%$& & \\
    Q5 & $\approx 3.5\%$&$\approx 0.35\%$ & & \\

    \bottomrule

    \end{tabular}
    \label{tbl:smartptrTime}
\end{table}


\section{Mathematische Formeln}

Formel \ref{form1} sollte jedem Bekannt vorkommen.

\begin{equation}
    e = m \cdot c^2
    \label{form1}
\end{equation}
\[ v = \frac{s}{t} \]
Und noch ein wenig Mathematik zeigt Formel \ref{form2}
\begin{equation}
    \ln(e) + \sin^2(p) + \cos^2 (p) = \sum_{n=0}^{\infty} \left(\frac{1}{2}\right)^n
    \label{form2}
\end{equation}

\section{Theorem}
\begin{definition}
Definition1
\end{definition}
\begin{definition}
Definition2
\end{definition}


\section{Bilder}
\Happy{} \Okay{} \Bad{}
Abbildung~\ref{pic:logo} zeigt das Logo in den Text eingebettet.

\begin{figure}[ht!]
    \centering
    \includegraphics[width = 0.4\textwidth]{images/logo}
    \caption{Logo }
    \label{pic:logo}
\end{figure}

\lipsum{1234}

Das ganze mit Tikz:
\begin{figure}
    \begin{center}
        \begin{tikzpicture}
            \tikzstyle{every entity}=[fill=blue!20,draw=blue,thick]
            \tikzstyle{every relationship}=[fill=orange!20,draw=orange,thick,aspect=1.5]

            \node[entity] (sheep) at (0,0) {Sheep};
            \node[entity] (genome) at (4,0) {Genome};
            \node[relationship] at (2,1.5) {has}
                edge (sheep)
                edge (genome);
            \draw[->] (sheep) -- (genome);
        \end{tikzpicture}
    \end{center}
    \label{pic:bla}

    \caption{bla}

\end{figure}




\section{Algorithmen}

\begin{algorithm}[H]
    \SetAlgoLined
    \KwData{this text}
    \KwResult{how to write algorithm with \LaTeX2e }
    initialization\;
    \While{not at end of this document}{
        read current\;
        \eIf{understand}{
            go to next section\;{}
            current section becomes this one\;{}
        }{
            go back to the beginning of current section\;{}
        }
    }
    \For{ i= 1 ,i<10,i++ } {
        bla(i)\;
    }
\caption{How to write }
\end{algorithm}


\section{Blindtext}

\subsection{blindtext package}
\blindtext[2]

\subsection{lipsum package}
Lipsum paragraphen 4-5 \\
\lipsum[4-5]

\subsection{text}

Hier steht richtig viel Text.
Hier steht richtig viel Text.
Hier steht richtig viel Text.
Hier steht richtig viel Text.
Hier steht richtig viel Text.
Hier steht richtig viel Text.
Hier steht richtig viel Text.
Hier steht richtig viel Text.
Hier steht richtig viel Text.
Hier steht richtig viel Text.
Hier steht richtig viel Text.
Hier steht richtig viel Text.

Hier steht richtig viel Text.
Hier steht richtig viel Text.
Hier steht richtig viel Text.
Hier steht richtig viel Text.
Hier steht richtig viel Text.
Hier steht richtig viel Text.
Hier steht richtig viel Text.
Hier steht richtig viel Text.
Hier steht richtig viel Text.
Hier steht richtig viel Text.
Hier steht richtig viel Text.
Hier steht richtig viel Text.
Hier steht richtig viel Text.
Hier steht richtig viel Text.
Hier steht richtig viel Text.
Hier steht richtig viel Text.
Hier steht richtig viel Text.
Hier steht richtig viel Text.
Hier steht richtig viel Text.
Hier steht richtig viel Text.

Hier steht richtig viel Text.
Hier steht richtig viel Text.
Hier steht richtig viel Text.
Hier steht richtig viel Text.
Hier steht richtig viel Text.
Hier steht richtig viel Text.
Hier steht richtig viel Text.
Hier steht richtig viel Text.
Hier steht richtig viel Text.
Hier steht richtig viel Text.
Hier steht richtig viel Text.
Hier steht richtig viel Text.
Hier steht richtig viel Text.
Hier steht richtig viel Text.
Hier steht richtig viel Text.
Hier steht richtig viel Text.
Hier steht richtig viel Text.
Hier steht richtig viel Text.
Hier steht richtig viel Text.
Hier steht richtig viel Text.

